\chapter{Limites}

% \begin{definition}[limit point]
%     Let \(\left(X,\tau\right)\) be a topological space, \(\symcal{B}\) a filter base on \(X\). We say \(x\) is a limit point of \(\symcal{B}\), if there exists a set \(E\subseteq X\), such that \(x\), as a limit point of \(E\), satisfies \[\emptyset\notin\eval{\symcal{B}}[E]\geqslant\symcal{N}\left(x\right).\]
% \end{definition}

% \begin{definition}[limit set]
%     Given \(f\colon S\to\left(X,\tau\right)\), let \(\Omega\coloneq\ker\overline{f\left(\symcal{B}\right)}\), then \[\forall_{x\in\Omega}\exists_{E}\left(\lim f\left(\set{B\cap E\in\symcal{P}\left(X\right)}[B\in\symcal{B}]\right)=x\right)\]
% \end{definition}

% \begin{definition}[derived set and closure]
%     The set of all limit points of a filter base \(\symcal{B}\) is called its limit set, denoted \(\symcal{B}'\).
% \end{definition}

\begin{definition}[limite]\label{definition:1.1}
    Soit \(f\) une application d'un ensemble \(S\) vers un espace topologique \(\left(X,\tau\right)\),\marginnote{Ici \(\tau\) est la topologie sur \(X\).} et \(\symcal{B}\) une base de filtre sur \(S\). Un point \(x\in X\) est appelé une \emph{limite}\index{limite!d'une fonction} de \(f\) sur \(\symcal{B}\), si pour tout
    \begin{equation}
        U\in\symcal{N}\left(x\right)\coloneq\set{U\in\tau}[U\ni x],\marginnote{Ici, nous disons que \(\symcal{N}\left(x\right)\) est le filtre de voisinage de \(x\), et \(U\) est un voisinage de \(x\).}
    \end{equation}
    il existe un \(B\in\symcal{B}\) tel que \(f\left(B\right)\subseteq U\). Dans ce cas, nous disons aussi que \(f\) \emph{tend} vers \(x\in X\) le long de \(\symcal{B}\), noté
    \begin{equation}
        \lim_{\symcal{B}}f=x,\qquad\lim f\left(\symcal{B}\right)=x,\qquad\text{ou}\qquad f\left(\symcal{B}\right)\to x.
    \end{equation}
    Symboliquement,
    \begin{equation}
        f\left(\symcal{B}\right)\to x\coloniff f\left(\symcal{B}\right)\geqslant\symcal{N}\left(x\right).
    \end{equation}
\end{definition}

% \begin{definition}
%     A sequence \(\set{x_n}_{n=0}^{+\infty}\) in a metric space \(\left(X,d\right)\) is said to \emph{converge}, if there exists a point \(x\in X\), such that for every \(\varepsilon>0\), there is an index \(N\), where \(d\left(x_n,x\right)<\varepsilon\) for all integers \(n>N\). In this case, we say that \(x\) is the \emph{limit}\index{limit!of a sequence} of \(\set{x_n}\), or that \(\set{x_n}\) converges to \(x\), written \(\lim_{n\to+\infty}x_n=x\) or simply \(x_n\to x\). Symbolically, \[\left(\lim_{n\to+\infty}x_n=x\right)\coloneq\forall_{\varepsilon\in\symbb{R}_{>0}}\exists_N\forall_{n\in\symbb{Z}_{\geqslant0}}\left(n>N\implies\left(x_n,x\right)<\varepsilon\right).\] A sequence that does not converge is said to \emph{diverge}.
% \end{definition}

\begin{example}[limit of a sequence]
    In \cref{definition:1.1}, let \(S=\symbb{N}\), then the function
    \begin{equation}
        f\colon\symbb{N}\to X,\quad n\mapsto x_n
    \end{equation}
    defines a sequence \(\set{x_n}_{n\in\symbb{N}}\) in the topological space \(\left(X,\tau\right)\). Now let
    \begin{equation}
        \symcal{B}=\left(\symbb{N}\ni n\to+\infty\right)\coloneq\set{\symbb{N}_{>N}}[N\in\symbb{N}].
    \end{equation}
    If for any neighborhood \(U\) of a point \(x\in X\), there exists some \(N\in\symbb{N}\) such that
    \begin{equation}
        n>N\implies x_n\in U,
    \end{equation}then \(x\) is called a \emph{limit}\index{limit!of a sequence} of \(\set{x_n}\). In this case, we also say that \(\set{x_n}\) \emph{converges} to \(x\), denoted by
    \begin{equation}
        \lim_{n\to+\infty}x_n=x\qquad\text{or}\qquad x_n\to x.
    \end{equation}
    A sequence that does not converge to any point is said to \emph{diverges}.
\end{example}

\begin{example}[limit of a real function]
    In \cref{definition:1.1}, let \(S\subseteq X=\symbb{R}\),
    \begin{align}
        \symcal{B}&=\left(t\to a\in\overline{\symbb{R}}\right)\\
        &\coloneq
        \begin{cases*}
            \set{\symbb{R}_{<M}}[M\in\symbb{R}] & if \(a=-\infty\), \\
            \set{\symbb{R}_{>M}}[M\in\symbb{R}] & if \(a=+\infty\), \\
            \set{\symbb{R}_{>a-\delta_1}\cap\symbb{R}_{<a+\delta_2}-\set{a}}[\left(\delta_1,\delta_2\right)\in\symbb{R}^2] & otherwise.
        \end{cases*}
    \end{align}
    If for any neighborhood \(U\) of a point \(A\in X\), there exists some \(V\in\symcal{B}\) such that
    \begin{equation}
        t\in V\implies f\left(t\right)\in U,
    \end{equation}then \(A\) is called a \emph{limit}\index{limit!of a real function} of \(f\left(t\right)\) as \(t\) tends to \(a\). In this case, we also say that \(f\left(t\right)\) \emph{tends} to \(A\) as \(t\) tends to \(a\), denoted by
    \begin{equation}
        \lim_{t\to a}f\left(t\right)=A.
    \end{equation}
\end{example}
